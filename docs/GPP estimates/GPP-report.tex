% Options for packages loaded elsewhere
\PassOptionsToPackage{unicode}{hyperref}
\PassOptionsToPackage{hyphens}{url}
%
\documentclass[
]{article}
\usepackage{amsmath,amssymb}
\usepackage{lmodern}
\usepackage{iftex}
\ifPDFTeX
  \usepackage[T1]{fontenc}
  \usepackage[utf8]{inputenc}
  \usepackage{textcomp} % provide euro and other symbols
\else % if luatex or xetex
  \usepackage{unicode-math}
  \defaultfontfeatures{Scale=MatchLowercase}
  \defaultfontfeatures[\rmfamily]{Ligatures=TeX,Scale=1}
\fi
% Use upquote if available, for straight quotes in verbatim environments
\IfFileExists{upquote.sty}{\usepackage{upquote}}{}
\IfFileExists{microtype.sty}{% use microtype if available
  \usepackage[]{microtype}
  \UseMicrotypeSet[protrusion]{basicmath} % disable protrusion for tt fonts
}{}
\makeatletter
\@ifundefined{KOMAClassName}{% if non-KOMA class
  \IfFileExists{parskip.sty}{%
    \usepackage{parskip}
  }{% else
    \setlength{\parindent}{0pt}
    \setlength{\parskip}{6pt plus 2pt minus 1pt}}
}{% if KOMA class
  \KOMAoptions{parskip=half}}
\makeatother
\usepackage{xcolor}
\usepackage[margin=1in]{geometry}
\usepackage{graphicx}
\makeatletter
\def\maxwidth{\ifdim\Gin@nat@width>\linewidth\linewidth\else\Gin@nat@width\fi}
\def\maxheight{\ifdim\Gin@nat@height>\textheight\textheight\else\Gin@nat@height\fi}
\makeatother
% Scale images if necessary, so that they will not overflow the page
% margins by default, and it is still possible to overwrite the defaults
% using explicit options in \includegraphics[width, height, ...]{}
\setkeys{Gin}{width=\maxwidth,height=\maxheight,keepaspectratio}
% Set default figure placement to htbp
\makeatletter
\def\fps@figure{htbp}
\makeatother
\setlength{\emergencystretch}{3em} % prevent overfull lines
\providecommand{\tightlist}{%
  \setlength{\itemsep}{0pt}\setlength{\parskip}{0pt}}
\setcounter{secnumdepth}{-\maxdimen} % remove section numbering
\newlength{\cslhangindent}
\setlength{\cslhangindent}{1.5em}
\newlength{\csllabelwidth}
\setlength{\csllabelwidth}{3em}
\newlength{\cslentryspacingunit} % times entry-spacing
\setlength{\cslentryspacingunit}{\parskip}
\newenvironment{CSLReferences}[2] % #1 hanging-ident, #2 entry spacing
 {% don't indent paragraphs
  \setlength{\parindent}{0pt}
  % turn on hanging indent if param 1 is 1
  \ifodd #1
  \let\oldpar\par
  \def\par{\hangindent=\cslhangindent\oldpar}
  \fi
  % set entry spacing
  \setlength{\parskip}{#2\cslentryspacingunit}
 }%
 {}
\usepackage{calc}
\newcommand{\CSLBlock}[1]{#1\hfill\break}
\newcommand{\CSLLeftMargin}[1]{\parbox[t]{\csllabelwidth}{#1}}
\newcommand{\CSLRightInline}[1]{\parbox[t]{\linewidth - \csllabelwidth}{#1}\break}
\newcommand{\CSLIndent}[1]{\hspace{\cslhangindent}#1}
\ifLuaTeX
  \usepackage{selnolig}  % disable illegal ligatures
\fi
\IfFileExists{bookmark.sty}{\usepackage{bookmark}}{\usepackage{hyperref}}
\IfFileExists{xurl.sty}{\usepackage{xurl}}{} % add URL line breaks if available
\urlstyle{same} % disable monospaced font for URLs
\hypersetup{
  pdftitle={GPP report},
  hidelinks,
  pdfcreator={LaTeX via pandoc}}

\title{GPP report}
\author{}
\date{\vspace{-2.5em}2022-09-01}

\begin{document}
\maketitle

\hypertarget{gpp-report}{%
\section{GPP report}\label{gpp-report}}

A quick look at the dissolved oxygen profiles for each of the tanks.
Clear diurnal patterns and obvious variation among tanks.

\includegraphics{GPP-report_files/figure-latex/DO profiles-1.pdf}

We estimated GPP, NPP, and ER based on the dawn-dusk-dawn
O\textsubscript{2} measurements. The calculations were taken from
Kritzberg et al. (2014) and can be summed up as:

\[ GPP = NPP + ER \]

\[ NPP = DO_{day2} - DO_{day1} \]

\[ ER = DO_{day1_{night}} - DO_{day2_{morning}} \]

Essentially, NPP is estimated by the difference in morning
O\textsubscript{2} from the first to second day. ER is estimated as the
loss of O\textsubscript{2} between dusk and dawn and is assumed to be
constant (a tenuous assumption which we may need to address). GPP is the
sum of NPP and ER. All measures are standardized to mg
O\textsubscript{2} L\textsuperscript{-1} hr\textsuperscript{-1}.

Some quick boxplots to observe the patterns among temperature and
nutrient treatments.

First up is gross production. Some interesting interactions seem to be
happening in the heated nutrient treatments. Seemingly, nutrient
addition have in inhibitory effect on total production. It will be
interesting to see how this plays out in future measurements.

\includegraphics{GPP-report_files/figure-latex/gpp boxplots-1.pdf}

Next, the ER estimates (measured as O\textsubscript{2} consumption)
mostly mirror the GPP patterns and suggest lower (less negative) rates
of respiration in the heated nutrient addition treatments.

\includegraphics{GPP-report_files/figure-latex/er boxplots-1.pdf}

Our estimates of net production, a proxy of biomass accumulation in the
mesocosms, shows a pattern opposite what I would predict. Namely, NPP
decreases with warming and nutrient additions. Initial thoughts on this
is that it could arise if warming and nutrient additions are revving up
the heterotrophic pathways. Ultimately, the systems are `closed', so ER
is limited by GPP and GPP may be limited by biomass--essentially the hi
temp-hi nüts treatments are substrate limited in both the production and
respiration sides of the equation.

\includegraphics{GPP-report_files/figure-latex/np boxplots-1.pdf}

The GPP:ER ratio shows the extent to which these systems may be
accumulating biomass (the should mirror closely the relative patterns of
NPP). Values above 1 represent systems that fix more C than they
respire, and values below 1 respire more C than they fix.
Unsurprisingly, the mesocosms are almost entirely above 1 since they are
pretty close to blank slates and respiration is dependent on C fixed
locally.

\includegraphics{GPP-report_files/figure-latex/NEP estimates-1.pdf}

\includegraphics{GPP-report_files/figure-latex/GPP-ER scatter-1.pdf}

When we look at Net production and ER, there is no clear relationship.

\includegraphics{GPP-report_files/figure-latex/NPP-ER scatter-1.pdf}

\hypertarget{metabolisms-estimates-from-modeled-continouous-o2-series}{%
\subsection{Metabolisms estimates from modeled continouous o2
series}\label{metabolisms-estimates-from-modeled-continouous-o2-series}}

Here we can use the time series to model the estimated diel signal in o2
and temperature and then, with a few assumptions, we can further
estimate air-water exchange and GPP, ER, and NEP.

Similar plots to above, GPP, ER, NEP and GPP-ER relationships

\includegraphics{GPP-report_files/figure-latex/cont gpp boxplots-1.pdf}

Next, the ER estimates (measured as O\textsubscript{2} consumption)
mostly mirror the GPP patterns and suggest lower (less negative) rates
of respiration in the heated nutrient addition treatments.

\includegraphics{GPP-report_files/figure-latex/cont er boxplots-1.pdf}

Our estimates of net production, a proxy of biomass accumulation in the
mesocosms, shows a pattern opposite what I would predict. Namely, NPP
decreases with warming and nutrient additions. Initial thoughts on this
is that it could arise if warming and nutrient additions are revving up
the heterotrophic pathways. Ultimately, the systems are `closed', so ER
is limited by GPP and GPP may be limited by biomass--essentially the hi
temp-hi nüts treatments are substrate limited in both the production and
respiration sides of the equation.

\includegraphics{GPP-report_files/figure-latex/cont np boxplots-1.pdf}

The GPP:ER ratio shows the extent to which these systems may be
accumulating biomass (the should mirror closely the relative patterns of
NPP). Values above 1 represent systems that fix more C than they
respire, and values below 1 respire more C than they fix.
Unsurprisingly, the mesocosms are almost entirely above 1 since they are
pretty close to blank slates and respiration is dependent on C fixed
locally.

\includegraphics{GPP-report_files/figure-latex/cont NEP estimates-1.pdf}

\includegraphics{GPP-report_files/figure-latex/contGPP-ER scatter-1.pdf}

When we look at Net production and ER, there is no clear relationship.

\includegraphics{GPP-report_files/figure-latex/cont NPP-ER scatter-1.pdf}

\hypertarget{compare-dawn-dusk-and-continous-methods}{%
\subsubsection{Compare Dawn-Dusk and continous
methods}\label{compare-dawn-dusk-and-continous-methods}}

\includegraphics{GPP-report_files/figure-latex/comb DD and Cont-1.pdf}

\hypertarget{takeaways}{%
\subsection{Takeaways}\label{takeaways}}

My initial takeaways from this are the data are a bit puzzling.
Somethings make a lot of sense: 1) tight relationships between GPP and
ER, 2) GPP:ER \textasciitilde\textasciitilde{} 1 so they are pretty much
in carbon balance, with just slightly higher production.

There are a number of things that throw me a loop: 1) The relative
patterns in NPP are opposite of what I would predict, which I touched on
a bit above.

\hypertarget{next-steps}{%
\subsection{Next steps}\label{next-steps}}

I am going to calculate metabolism based on the entire diel
O\textsubscript{2} cycle rather than the three time points. This include
estimating air-water exchange. It is possible this term is more
important than we assumed and we are underestimating fluxes in the
warmed mesocosms. If the air-water flux is large and the underestimation
is systematic with productivity, this could account for the opposing
relative patterns in NPP estimates across treatments.

Or, as I noted earlier, it could be that the warming and nutrient
additions are really stimulating the breakdown of C and there is a super
tight C cycle going down.

Jeff--how does this jive with the initial patterns you are seeing in
emergence? Do the heated+/nutrient+ tanks have lower emergence?

Would love to hear y'all's thoughts or questions on any of this.

\hypertarget{references}{%
\section*{References}\label{references}}
\addcontentsline{toc}{section}{References}

\hypertarget{refs}{}
\begin{CSLReferences}{1}{0}
\leavevmode\vadjust pre{\hypertarget{ref-kritzberg2014}{}}%
Kritzberg, E. S., W. Granéli, J. Björk, C. Brönmark, P. Hallgren, A.
Nicolle, A. Persson, and L.-A. Hansson. 2014.
\href{https://doi.org/10.1111/fwb.12267}{Warming and browning of lakes:
Consequences for pelagic carbon metabolism and sediment delivery}.
Freshwater Biology 59:325--336.

\end{CSLReferences}

\end{document}
